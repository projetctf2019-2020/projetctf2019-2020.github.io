\chapter*{Conclusion}
\label{chap:Mini Projet}
\addcontentsline{toc}{chapter}{Conclusion}
\markboth{Conclusion}{Conclusion}

Au cours de ces trois premiers mois consacrés à notre projet CTF, nous avons eu l'occasion d'effectuer de nombreuses recherches afin de comprendre ce qu'était réellement un CTF.
Nous avons pu constater qu'il existe de nombreux domaines dans lesquels peuvent se dérouler des CTFs, ce qui implique ainsi une multitude d'outils que nous avons dû sélectionner pour affiner notre approche sur le sujet. En effet, tous ces outils sont plus ou moins complexes et proposent pour la plupart une grande variété d'options ou de modes d'utilisations.
Avec l'aide de Kali Linux, nous avons donc pu apprendre à manier les principaux outils, que nous considérons comme les bases pour réaliser un CTF (Nmap, Nitko, Dirbuster, Metasploit, etc...).
Grâce à cet apprentissage, nous avons pu mettre en pratique ces connaissances sur des CTFs.
Il nous faudra étudier plus précisément et coupler ces outils fondamentaux à de nouveaux afin d'élargir nos capacités de Pentest.