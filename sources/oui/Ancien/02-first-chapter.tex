\chapter*{Introduction}
\label{chap:Mini Projet}
\addcontentsline{toc}{chapter}{Introduction}
\markboth{Conclusion}{Conclusion}
\label{sec:conclusion}
Au cours de notre deuxième année de DUT en réseaux et télécommunications, nous avons réalisé un projet en groupe tutoré par M. GUILLEMIN ainsi que M. CHEVALLIER. Notre projet, qui a pour but d’augmenter notre autonomie et notre esprit de recherche face à une tâche complexe, a été orienté vers la sécurité informatique. En effet, notre sujet « Capture the flag » ou plus couramment appelé CTF, est un exercice d’infiltration système qui permet de vérifier la sécurité d’un service informatique. Le projet CTF a été mis en place en Septembre 2019. Nous n’avons donc reçu aucune base de nos aînés, ce qui va impliquer un rapport contenant majoritairement de la documentation à propos des outils d’infiltrations présent sur la distribution Kali Linux.\\
Sachant que le sujet est très vaste, nous allons essayer de nous focaliser sur des attaques de serveurs Web afin de pouvoir complètement traiter la question.\\

Avant de commencer à lire ce rapport, il est essentiel de savoir que tout ce qui y est répertorié ne doit en aucun cas être utilisé contre un système sans l’autorisation de son propriétaire au risque de lourdes peines.



%\begin{figure}[htp]
%  \centering
 % \input{images/tikz_diagram}
  %\caption{Exemple de diagramme TikZ.}
  %\label{fig:une-image}
%\end{figure}

%%% Local Variables: 
%%% mode: latex
%%% TeX-master: "isae-report-template"
%%% End: 