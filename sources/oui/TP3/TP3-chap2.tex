\subsection{Début du CTF}
\label{chap:Mini Projet}

\subsubsection{Récolte d'informations}
Si ce n'est pas déjà fait, allumez la machine \textbf{TP3-CTF}.

\begin{enumerate}
    \item Effectuez un scan sur votre réseau pour trouver l'IP de la machine à attaquer.\\
        Quelle commande faut-t'il effectuer? Indiquez la commande ainsi que l'IP de la machine cible.\\
        
    Il faut maintenant récupérer des informations sur la machine. Pour ce faire, il faut utiliser un outil de scan de ports et de services.\\
    
    \item Faites un scan de ports et des services sur la machine. Quel outil faut-il utiliser? Scannez maitenant les versions des services. Quel option faut-il ajouter? Listez le nom des services ainsi que les versions que vous avez trouvé.\\
    
    Une fois fait, vous devriez constater que cette machine héberge un serveur web.\\
    
    \item Expliquez à partir du cours les méthodes d'attaque sur un "CTF web". \\
    
    \item Utilisez un outil vu en cours pour scanner les fichiers du site web. Quel est cet outil? Selon-vous, quelle serait la méthode pour attaquer le site à partir des informations que vous venez de trouver? Faites un schéma de l'attaque.
\end{enumerate}

\subsubsection{Exploitation des failles}
