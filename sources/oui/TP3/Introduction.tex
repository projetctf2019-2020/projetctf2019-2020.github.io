\chapter*{Introduction}
\addcontentsline{toc}{chapter}{Introduction}
\markboth{Introduction}{Introduction}

La sécurité informatique au sein d’une entreprise est devenue le domaine avec le plus grand enjeux. Il faut donc du personnel spécialisé dans ce domaine afin de la mettre en place. On se rend facilement compte que le meilleur moyen de s’améliorer dans ce milieu est dans un premier temps de se documenter puis de réaliser des attaques. C’est à ce moment-là que le "Capture The Flag" ou bien "Capturer Le Drapeau" intervient. A l’origine, un CTF est un jeu à l’air libre où deux équipes s’affrontent pour s’emparer du drapeau de l’adversaire. On peut alors s’apercevoir que le monde informatique est semblable à celui réel. En effet, notre CTF a pour but d’infiltrer une machine cible et de trouver un document, le drapeau, en toute légalité. Le CTF s’est démocratisé en 1996 lors des premières compétitions organisées par la DEF CON. La DEF CON est la convention de hackeur la plus connue du monde. \\
Les CTF s’inspirent de la vraie vie même si cela reste un terrain d'entraînement. Les CTF reposent sur plusieurs domaines qui sont : le reverse engineering, l’exploitation web, le forensic, le réseau, la cryptographie, la sécurité mobile et la stéganographie.
Tous ces domaines sont les piliers de la sécurité informatique. Il faudra donc être polyvalent afin d’exploiter les failles et de résoudre un CTF. Nous allons donc voir au cours de ce rapports différents moyens de parvenir à nos fins.