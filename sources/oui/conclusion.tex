\chapter{Conclusion}

Au cours de ce module consacré au CTF, nous avons eu l'occasion d'effectuer de nombreuses recherches afin de comprendre ce qu'était réellement un CTF.
Nous avons pu constater qu'il existe de nombreux domaines dans lesquels peuvent se dérouler des CTFs, ce qui implique ainsi une multitude d'outils qui ont été sélectionnés pour affiner notre approche sur le sujet. En effet, tous ces outils sont plus ou moins complexes et proposent pour la plupart une grande variété d'options ou de modes d'utilisations.
Avec l'aide de Kali Linux, nous avons donc pu apprendre à manier les principaux outils, que nous considérons comme les bases pour réaliser un CTF (Nmap, Nitko, Dirbuster, Metasploit, etc...). Il faut tout de même se rappeler que ce n'est pas la distribution qui permet le pentesting mais les outils qu'elle contient.